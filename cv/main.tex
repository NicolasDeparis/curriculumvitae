\documentclass[11pt,a4paper,sans]{moderncv}
\moderncvstyle{banking}
\moderncvcolor{burgundy}
\usepackage[utf8]{inputenc}
\usepackage[english]{babel}


%\definecolor{color1}{rgb}{015,0.65,0}% primary scheme color


\usepackage[scale=0.84]{geometry}
\thispagestyle{empty}


\newcommand{\emma}{{\texttt{EMMA}}}
\newcommand{\aton}{{\texttt{ATON}}}
 
%\makeatletter
%\@ifpackageloaded{moderncvstylebanking}{%
%\let\oldmakecvtitle\makecvtitle
%%\renewcommand*{\makecvtitle}{%
%%  {\centering\framebox{\includegraphics[width=\@photowidth]{\@photo}}\par\vspace{10pt}}%
%%  \oldmakecvtitle%
%%}%
%%}
%\makeatother

%\setlength{\topmargin}{15pt}
%
%\makeatletter
%\@ifpackageloaded{moderncvstylebanking}{%
%\let\oldmakecvtitle\makecvtitle
%\renewcommand*{\makecvtitle}{%
%  {\par\vspace{100pt}}%
%  \oldmakecvtitle%
%  \par\vspace{30pt}%
%}%
%}
%\makeatother

%----------------------------------------------------------------------------------------
%	NAME AND CONTACT INFORMATION SECTION
%----------------------------------------------------------------------------------------

\firstname{Nicolas}
\familyname{Deparis} 

\title{Research engineer}
\address{11 rue de l'universite}{67000 Strasbourg, France}
\mobile{+33~(0)6~07~49~65~85}
%\phone{(000) 111 1112}
%\fax{(000) 111 1113}
\email{nicolas.deparis@astro.unistra.fr}
\homepage{github.com/NicolasDeparis}
\extrainfo{DOB 24/05/1986 - 33 years old}

%\quote{}

\usepackage{enumitem}
\usepackage{amssymb}
%----------------------------------------------------------------------------------------

\begin{document}



\makecvtitle


%\section{Research interests}
%
%%I am currently working with radiative hydrodynamical simulations in the cosmological context of the Epoch of Reionization.
%%I am focused on star formation and feedback processes -- radiation, stellar wind, supernovae --
%%at high redshift, and how theses processes influence galaxy formation history.
%%I am specially motivated by the field of high performance computing and software engineering.
%
%\begin{itemize}[label=--,leftmargin=1cm ,parsep=0cm,itemsep=0cm,topsep=0cm]
%%\item Cosmology - Reionization
%%\item Galaxy formation and evolution
%\item Star formation and feedback
%\item High performance computing
%\item Data visualization
%\end{itemize}

%\vspace{30pt}

%----------------------------------------------------------------------------------------
%	WORK EXPERIENCE SECTION
%----------------------------------------------------------------------------------------
\section{Experience}

\cventry
{Observatoire Astronomique de Strasbourg/CNRS}
{Scientific software developer}
{Fixed-term contract}
{2018-2020}
{}
{}
{Develloping the SVOM MXT Scientific pipeline using microservice architecture}


\cventry
{Observatoire Astronomique de Strasbourg/CNRS}
{6 months}
{Fixed-term contract}
{2014}
{}
{}
{Implementing stellar formation in a RHD cosmological code}

%\cventry
%{Observatoire Astronomique de Strasbourg}
%{3 months}
%{Master Internship}
%{2011}
%{}
%{}
%{Study of the cosmological tidal field with N-body simulation. %\newline{}
%Programming a multigrid algorithm to solve Poisson equation on GPU}

%\cventry
%{2010}
%{4 weeks }
%{Master 1 training}
%{Observatoire Astronomique de Strasbourg}
%{}
%{}
%{Study of the Epoch of Reionization with radiative transfer simulation. \newline 
%Integrating multi-wavelength capability in the \aton code}

%\cventry{2007}{Stage de 10 semaines}{Espace Info Énergie de Saint Dié des Vosges -- avec Phillipe Mangenot}{Étude énergétique du bâtiment du Centre Permanent d'Initiative à l’Environnement de Lusse}{Modélisation thermique du bâtiment, relevé et étude des points de consommations (eau, gaz, électricité), proposition de solutions}{}


%----------------------------------------------------------------------------------------
%	EDUCATION SECTION
%----------------------------------------------------------------------------------------
\section{Education}

\cventry
{Strasbourg University}
{Thesis title : Numerical study of the reionization with the simulation code EMMA}
{Ph.D. in Astrophysics}
{2014--2017}
{}
{Supervisor: Dominique Aubert}


\cventry
{Strasbourg University}
{Specialization : Astrophysics}
{Master in Theoretical Physics}
{2009--2011}
{}
{}

\cventry
{Louis Pasteur University}
{Specialization : Géophysics}
{Licence "Science de la Terre de l'Univers et de l’Environnement"}
{2008--2009}
{}
{}


\cventry
{Paul Verlaine University}
{Specialization : Techniques Instrumentales}
{Diplôme Universitaire de Technologie "Mesures Physiques"}
{2005--2007}
{}
{}


%\cventry{2004--2005}{Diplôme d’Études Universitaires Générales}{Nancy 2 University}{Nancy}{Non validé}{Option :  Psychologie -- Unité d'Enseignement : Philosophie}

%\cventry{2002--2004}{Baccalauréat}{Lycée George Baumont}{Saint Dié des Vosges}{}{Option : Sciences et Technologies Industrielles -- Spécialité : Électronique}




% %----------------------------------------------------------------------------------------
% %	COMPUTER SKILLS SECTION
% %----------------------------------------------------------------------------------------

% \section{Compétences Informatiques}

% \cvitem{Languages}{\textsc{C},\textsc{C}\texttt{++},\textsc{JAVA},\textsc{PYTHON},\textsc{FORTRAN},\textsc{BASH},\textsc{PERL},\LaTeX , \textsc{CUDA}, \textsc{MySQL}}

% \cvitem{Logiciels}{Microsoft Office, LibreOffice, Photoshop, GIMP, Eclipse}
% \cvitem{Systemes}{Windows (XP, Vista, Seven), Linux (Ubuntu, Android)}

% %----------------------------------------------------------------------------------------
% %	DIVERS SECTION
% %----------------------------------------------------------------------------------------

% \section{Divers}

% \cvitemwithcomment{Anglais}{lu, écrit, parlé}{(séjour de 5 mois en Nouvelle Zélande)}
% \cvitemwithcomment{Permis}{B}{}

% %----------------------------------------------------------------------------------------
% %	INTERESTS SECTION
% %----------------------------------------------------------------------------------------

% \section{Centres d'intérêts}

% %\renewcommand{\listitemsymbol}{-~} % Changes the symbol used for lists

% \cvlistdoubleitem{Sciences}{Cuisine : Boulangerie, Brasserie}
% \cvlistdoubleitem{Cinéma}{Sport : Velo, Snowboard, Jonglage}
% \cvlistdoubleitem{Voyages}{Photographie}







\section*{Schools}

\cvitem{11/2018}{IV ASTERICS VO School - Observatoire de Strasbourg, France}

\cvitem{11/2016}{Parallel computing by David Brusson - Ecole Supérieure du professorat et de l'éducation, Strasbourg - France}
\cvitem{06/2016}{Gutenberg School on Astrophysics - Stars and Galaxy Formation - Observatoire de Strasbourg, France}
\cvitem{05/2016}{Galaxy formation and evolution in a cosmological context by Andrea Cattaneo - Institut d'Astrophysique de Paris, France}
\cvitem{01/2016}{From BioImage Processing to BioImage Informatics - Télécom Physique Strasbourg France}
\cvitem{12/2015}{Principle of imaging for membrane systems - Institut Charles Sadron, Cronenbourg, Strasbourg, France}
\cvitem{03/2015}{Numerical Simulations in Astrophysics - Observatoire de Strasbourg, France}

\section{Computing skills}

\cvitem{Languages}{Python, C/C++, Fortran, Java}
\cvitem{Libraries}{MPI,CUDA,OpenMP,HDF5,OpenGL}
\cvitem{Tools}{Git, docker, Valgrind}
%\cvitem{Others}{Photoshop, Blender, Unreal Engine 4}
\cvitem{Known HPC centers}{PRACE TGCC Curie (France), PRACE CINES Occigen (France), OLCF Titan (USA) }


%\vspace{30pt}

\section*{Conferences}
\cvitem{06/2016}{Illuminating the Dark Ages: Quasars and Galaxies in the Reionization Epoch - MPIA Summer Conference 2016-  Heidelberg, Germany}
\cvitem{06/2016}{Presentation at Journées de la SF2A - Lyon, France}
\cvitem{04/2016}{Presentation at 13th Potsdam/AIP Thinkshop “Near Field Cosmology” - Obergurgl, Tyrol, Austria}
\cvitem{10/2015}{Presentation at meeting ORAGE - Roscoff, France}
\cvitem{05/2015}{Poster at The Olympian Symposium 2015 Cosmology and the Epoch of Reionization - Paralia Katerini's, Mount Olympus, Greece}
\cvitem{05/2015}{CLUES meeting 2015 - Copenhague, Danemark}
%\cvitem{12/2014}{ Kick-off ORAGE / Paris, France}


%\section*{Outreach}
%\cvitem{06/2015}{Kids university - Strasbourg, France}

\nocite{*}
\renewcommand\refname{Publications}
\bibliographystyle{ieeetr}
\bibliography{biblio.bib}


\section*{References}

\begin{tabular}{lr}
\begin{minipage}[t]{3.5in}
Dr. Dominique Aubert\\
Observatoire Astronomique de Strasbourg\\
11 rue de l'Université\\
67000 Strasbourg \\
France\\
\phonesymbol\ +33 (0) 3 68 85 24 68\\
\emailsymbol\ \href{mailto:dominique.aubert@unistra.fr}{dominique.aubert\textrm{@}unistra.fr}
\end{minipage}
&
\begin{minipage}[t]{4in}
Dr. Pierre Ocvirk\\
Observatoire Astronomique de Strasbourg\\
11 rue de l'Université\\
67000 Strasbourg\\
France\\
\phonesymbol\ +33 (0) 3 68 85 24 40\\
\emailsymbol\ \href{mailto:pierre.ocvirk@astro.unistra.fr}{pierre.ocvirk\textrm{@}astro.unistra.fr}
\end{minipage}

\end{tabular}


\clearpage

%----------------------------------------------------------------------------------------
%	COVER LETTER
%----------------------------------------------------------------------------------------
%
%%
%%
%%\recipient{CCI}{Dispositif DRIV' Emploi} 
%%\date{Provenchères sur Fave, le \today} 
%%\opening{Madame, Monsieur, }
%%\closing{
%%Conscient que ce seul courrier ne peut refléter l'étendue de mes compétences, je vous propose de nous rencontrer à la date de votre convenance. \\ ~~\\
%%Dans cette attente, je vous prie de croire, Madame, Monsieur, à l'assurance de toute ma considération.} 
%%\enclosure[Pieces jointes]{Curriculum Vit\ae{}} 
%%
%%\makelettertitle 
%%
%%
%%Désireux d'obtenir une première expérience dans le domaine du développement informatique je pense convenir au profil que vous recherchez.
%
%Issu d'une formation de Master d'Astrophysique, je dispose de connaissances en manipulation de bases de données volumineuses (catalogue d'étoiles par exemple). Je suis également habitué au traitement et à l'analyse de grandes quantités de données de manière automatisée.
%
%
%
%
%%Issu d'une formation de Master de Physique et conscient des immenses possibilités offertes par la médecine numérique, dans laquelle votre société fait figure de pionnière, c'est avec enthousiasme que je soumets ma candidature pour l'emploi de développeur \textsc{C}/\textsc{C}\texttt{++} que vous proposez.
%%
%%
%%Lors de ma formation j'ai été amené à participer au développement de simulations physiques tridimensionnelles de grande envergure ainsi qu'à l'analyse et à la représentation graphique des résultats. Passionné par le calcul scientifique, je possède de bonnes connaissances dans le domaine du calcul parallèle et dans le traitement et l'analyse de grande quantité d'informations 3D mais je dispose également de notions dans la programmation d'interface graphique et dans la gestion de bases de données.
%
%
%
%


%
%\recipient{École doctorale physique et chimie-physique}{ED 182} 
%\date{Strasbourg, le \today} 
%\opening{Madame, Monsieur, }
%\closing{Je vous prie de croire, Madame, Monsieur, à l'assurance de toute ma considération.} 
%%\enclosure[Pièces jointes]{Curriculum Vit\ae{}} 
%
%\makelettertitle
%
%
%
%Titulaire d'un Master en Astrophysique, je souhaite prolonger mes études, jusqu'au grade de Docteur, au sein de l'Observatoire Astronomique de Strasbourg.
%
%Le sujet sur lequel j'ai travaillé, et sur lequel j'aimerais poursuivre, porte sur l'étude, à l'aide de méthodes numériques, de l'époque de réionisation, époque correspondant à l'apparition des premières étoiles dans l'Univers. Dans un futur proche de nouvelles missions d'observations (telles JWST en 2018 ou SKA et ATHENA en 2028) seront suffisamment puissantes, pour être en mesure de voir la naissance de ces premières étoiles.
%
%Les études, comme celles que je compte mener grâce à votre soutien, ont pour but d'anticiper, de façon théorique, sur les résultats potentiellement fournis par ces observations. Elles permettront ainsi de poser de meilleures contraintes, non seulement sur les futurs instruments, mais surtout sur la physique générale des processus impliqués à cette époque.
%
%
%Je réalise actuellement ma troisième collaboration avec Dominique Aubert. Le code sur lequel nous travaillons, calcule l’évolution de différentes quantités physiques, telles que la matière noire, le gaz, le rayonnement, ou les étoiles.
%
%%Lors de mon stage de Licence, j'ai participé au développement de la partie radiative en y implémentant une capacité multi longueur d'onde. Je l'ai ensuite utilisée pour déterminer l'impacte du préchauffage du milieu interstellaire par les photons les plus énergétiques dans le processus de réionisation. 
%
%Lors de mon stage de Licence, j'ai participé au développement de la partie radiative. Je l'ai ensuite utilisée pour déterminer l'impact des photons les plus énergétiques dans le processus de réionisation. 
%Mon stage de Master s'est concentré sur la partie matière noire (aka N-Body). J'ai optimisé la résolution de l'équation de Poisson en implémentant une méthode dite multigrille.
%%Pour obtenir des performances optimales, ce code tire partie des capacités de calcul parallèle des cartes graphiques (GPGPU avec NVIDIA CUDA).
%J'ai ensuite analysé le potentiel obtenu pour étudier l'influence du champ de marée sur la formation des grandes structures cosmologiques.
%Actuellement, je travaille à implémenter la formation stellaire, une partie qui interagit avec tous les processus physiques du code. 
%%Lors-ce qu'une particule stellaire est crée, une partie du gaz (partie hydrodynamique) est convertie en particule (partie N-body), puis émet des rayons ionisant (partie rayonnement) et en fin de vie, explose en supernovæ (partie hydrodynamique et chimie).
%Ces stages sous sa tutelle m’ont permis d’acquérir un solide bagage dans le domaine du calcul hautes performances, ainsi qu'une bonne vision d'ensemble du domaine des simulations cosmologiques.
%
%Cette thèse de Doctorat se pose donc clairement dans la continuité de ce que j'ai déjà accompli par le passé. Tant par les connaissances déjà acquises sur le sujet que par les nombreux outils logiciels déjà développés, mais aussi, et surtout, sur les nombreuses problématiques et interrogations soulevées.
%
%J'ai un fort intérêt pour le développement de simulations numériques et pour le calcul scientifique en général. L'accès à des ressources matérielles telles que celles disponibles ici est pour moi une réelle opportunité.
%De plus, utiliser ces ressources pour l'étude des propriétés intrinsèques de  l'Univers, au niveau de ses plus grandes échelles, me motive particulièrement. Cette thèse me permet de réunir deux de mes passions que sont l'informatique et la cosmologie, et m'ouvre des portes sur la possibilité que peut-être un jour je participerai à l’évolution du paradigme commun. De plus, le fait d'avoir d'ores et déjà développé un bon relationnel au sein de l'équipe et d'être familier avec l'environnement de l'observatoire est pour moi une motivation supplémentaire.
%
%Conscient que ce seul courrier ne peut refléter l'étendue de l'intérêt que je porte à ce domaine, je reste dans l'attente de vous rencontrer, pour vous le présenter plus en détail.










% \recipient{Équipe Imabio}{Institut Pluridisciplinaire Hubert Curien} 
% \date{Strasbourg, le \today} 
% \opening{Madame, Monsieur, }
% \closing{Je vous prie de croire, Madame, Monsieur, à l'assurance de toute ma considération.} 
% \enclosure[Pièces jointes]{Curriculum Vit\ae{}} 

% \makelettertitle

% %Le domaine de e la médecine 

% Aujourd'hui, d'innombrables domaines ont besoin d’énormément de puissance de calcul pour résoudre des problèmes toujours plus complexe. 
% Je me tourne vers vous car le fond de vos recherches me parait extrêmement louable, et participer au développement d'un champ aussi essentiel que celui de l'imagerie clinique serait pour moi une véritable aubaine.


% Titulaire d'un master en astrophysique, je suis actuellement contractuel à l'observatoire astronomique de Strasbourg en tant que développeur d'application. J'assiste Dominique Aubert dans le développement d'un code de simulation cosmologique. Travailler avec lui sur ce type de code m'a permis d'obtenir de solides connaissances dans le domaine du calcul scientifique hautes performances tant sur la partie haut niveau (utilisation de grilles AMR, mise en place d'un solveur d’équation différentielles par multigrille, ...), que bas niveau (parallélisation par MPI, openMP et GPGPU, utilisation du centre de calcul HPC, optimisations mémoire, ...).

% De part ma formation, je dispose également d'une bonne vue d'ensemble du domaine de la physique, mais aussi de celui de l'instrumentation. Je pense être en mesure de vous apporter toutes les compétences nécessaires pour mener à bien cette mission.


%Je dispose également d'un certain bagage en physique ainsi que de bonnes capacités relationnelles.











% Conscient que ce seul courrier ne peut refléter l'étendue de l'intérêt que je porte au domaine du calcul scientifique, je reste dans l'attente de vous rencontrer, à la date de votre convenance.

% \makeletterclosing


%----------------------------------------------------------------------------------------




\recipient{Équipe Imabio}{Institut Pluridisciplinaire Hubert Curien} 
\date{Strasbourg, le \today} 
\opening{Madame, Monsieur, }
\closing{Je vous prie de croire, Madame, Monsieur, à l'assurance de toute ma considération.} 
\enclosure[Pièces jointes]{Curriculum Vit\ae{}} 

%\makelettertitle

% %Le domaine de e la médecine 

% Aujourd'hui, d'innombrables domaines ont besoin d’énormément de puissance de calcul pour résoudre des problèmes toujours plus complexe. 
% Je me tourne vers vous car le fond de vos recherches me parait extrêmement louable, et participer au développement d'un champ aussi essentiel que celui de l'imagerie clinique serait pour moi une véritable aubaine.


% Titulaire d'un master en astrophysique, je suis actuellement contractuel à l'observatoire astronomique de Strasbourg en tant que développeur d'application. J'assiste Dominique Aubert dans le développement d'un code de simulation cosmologique. Travailler avec lui sur ce type de code m'a permis d'obtenir de solides connaissances dans le domaine du calcul scientifique hautes performances tant sur la partie haut niveau (utilisation de grilles AMR, mise en place d'un solveur d’équation différentielles par multigrille, ...), que bas niveau (parallélisation par MPI, openMP et GPGPU, utilisation du centre de calcul HPC, optimisations mémoire, ...).

% De part ma formation, je dispose également d'une bonne vue d'ensemble du domaine de la physique, mais aussi de celui de l'instrumentation. Je pense être en mesure de vous apporter toutes les compétences nécessaires pour mener à bien cette mission.


%Je dispose également d'un certain bagage en physique ainsi que de bonnes capacités relationnelles.











% Conscient que ce seul courrier ne peut refléter l'étendue de l'intérêt que je porte au domaine du calcul scientifique, je reste dans l'attente de vous rencontrer, à la date de votre convenance.

% \makeletterclosing


\end{document}